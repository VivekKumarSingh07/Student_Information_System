\chapter{Implementation}

\section{FlaskApp - Python}

\begin{python}
from flask import Flask, url_for, render_template, g, request, \
    redirect, flash, session, abort
from flask_login import LoginManager
from wtforms import Form, TextField, TextAreaField, validators,  \
    StringField, SubmitField
import os
import sqlite3
from urllib.parse import unquote
from jinja2 import Template

app = Flask(__name__)
DATABASE = "data.db"
# Config
app.config.from_object(__name__)

ID = None

def average(t1, t2, t3):
    return int((int(t1) + int(t2) + int(t3))//3)

def get_db():
    db = getattr(g, '_database', None)
    if db is None:
        db = g._database = sqlite3.connect(DATABASE)
        db.create_function('avg', 3, average)
        db.row_factory = sqlite3.Row
    return db


def query_db(query: object, args: object, one: object = False) -> object:
    cur = get_db().execute(query, args)
    rv = cur.fetchall()
    cur.close()
    return (rv[0] if rv else None) if one else rv


def change_db(query, args=()):
    cur = get_db().execute(query, args)
    get_db().commit()
    cur.close()

@app.teardown_appcontext
def close_connection(exception):
    db = getattr(g, '_database', None)
    if db is not None:
        db.close()


@app.route("/")
def index():
    if not session.get('logged_in'):
        return login()
    else:
        return render_template("index.html", id=ID)


@app.route("/login", methods=['GET', 'POST'])
def login():
    if request.method == 'GET':
        return render_template("login.html", error=False)
    if request.method == 'POST':
        try:
            password = request.form['password']
            teacher_id = int(request.form['teacher_id'])
            print(password, teacher_id)
        except ValueError:
            return render_template("login.html", error=True)

        teacher = query_db("SELECT * FROM Teacher \
                                WHERE teacher_id = ?",
                           [teacher_id],  one=True)
        if teacher is None:
            return render_template("login.html", error=True)
            # TODO
        if teacher["password"] == password:
            session['logged_in'] = True
            global ID
            ID = teacher_id
            return index()
        else:
            return render_template("login.html", error=True)


@app.route('/logout')
def logout():
    session.pop('logged_in', None)
    flash('You were logged out')
    return redirect(url_for('index'))


@app.route('/view', methods=['GET', 'POST'])
def view():
    global ID
    teacher = query_db("SELECT * FROM Teacher \
                         WHERE teacher_id=?", [ID], one=True)
    student_list = query_db("SELECT * FROM Student \
                             WHERE semester IN (SELECT semester FROM Subject, \
                             Teacher WHERE Teacher.sub_code=Subject.sub_code AND teacher_id=?)", [ID])
    if request.method == 'GET':
        if not session.get('logged_in'):
            return login()
        else:
            return render_template("user.html", id=ID, teacher=teacher, error=None, student_list=student_list)
    if request.method == 'POST':
        old = request.form['oldPassword']
        new = request.form['newPassword']

        if teacher['password'] == old:
            change_db(
                "UPDATE Teacher SET password = ? WHERE teacher_id = ?", (new, ID))
            return render_template("user.html", id=ID,
                                   teacher=teacher, changed=True,  student_list=student_list)
        else:
            return render_template("user.html", id=ID,
                                   teacher=teacher, error=True,  student_list=student_list)


@app.route('/teachers')
def teachers():
    global ID
    print(ID)
    teacher_list = query_db("SELECT * FROM TEACHER", [])
    return render_template("/teachers_info.html", id=ID,
                           teacher_list=teacher_list)


@app.route('/students')
def students():
    global ID
    student_list = query_db("SELECT * FROM Student", [])
    return render_template("/students_info.html", id=ID,
                           student_list=student_list)


@app.route('/modify/<string:entity>/<string:uid>/<string:uid2>',
           methods=['GET', 'POST'])
def modify(uid, uid2, entity):
    global ID
    teacher_list = query_db("SELECT * FROM TEACHER", [])
    if entity == "teacher":
        values = query_db("SELECT * FROM Teacher \
                            WHERE teacher_id=?", [uid], one=True)
        if request.method == 'GET':
            return render_template("modify.html", id=ID,
                                   entity="Teacher",  identity=values)
        if request.method == 'POST':
            data = request.form.to_dict()
            dic = [data['teacher_id'], data['sub_code'],
                    data['teacher_name'], data['phone'], uid]
            change_db(
                "UPDATE Teacher SET teacher_id=?, sub_code=?,  "
                "teacher_name=?, phone=? WHERE teacher_id=?", dic)
            return logout()
    if entity == "student":
        values = query_db("SELECT * FROM Student \
                            WHERE student_id=?", [uid], one=True)
        print(values['student_name'])
        if request.method == 'GET':
            return render_template("modify.html", id=ID,
                                   entity="Student", identity=values)
        if request.method == 'POST':
            data = request.form.to_dict()
            print(data.keys())
            dic = [data['student_id'], data['student_name'],
                    data['academic_year'], data['branch_code'], uid]
            change_db(
                "UPDATE Student SET student_id=?, student_name=?,  "
                "academic_year=?, branch_code=? WHERE student_id=?", dic)
            return logout()
    if entity == "attendance":
        values = query_db("SELECT * FROM Attendance \
                            WHERE sub_code=? AND student_id=?", [uid2, uid], one=True)
        if request.method == 'GET':
            return render_template("modify.html", id=ID,
                                   entity="Attendance", identity=values)
        if request.method == 'POST':
            data = request.form.to_dict()
            print(data.keys())
            #TODO
            # avg = int((int(data['a1']) + int(data['a2']) + int(data['a3']))/3)
            dic = [data['sub_code'], data['student_id'],
                    data['a1'], data['a2'], data['a3'], data['a1'], data['a2'], data['a3'], uid2, uid]
            #STORED_PROCEDURE_DEMO
            change_db(
                "UPDATE Attendance SET sub_code=?, student_id=?, "
                "a1=?, a2=?, a3=?, final_attendance=avg(?, ?, ?)  WHERE sub_code=? AND student_id=?", dic)
            return redirect(url_for("attendance"))
    if entity == "marksheet":
        values = query_db("SELECT * FROM Marksheet \
                            WHERE sub_code=? AND student_id=?",
                          [uid2, uid], one=True)
        if request.method == 'GET':
            return render_template("modify.html", id=ID,
                                   entity="Marksheet", identity=values)
        if request.method == 'POST':
            data = request.form.to_dict()
            print(data.keys())
            #TODO
            # avg = int((int(data['a1']) + int(data['a2']) + int(data['a3']))/3)
            dic = [data['sub_code'], data['student_id'],
                    data['t1'], data['t2'], data['t3'],
                   data['t1'], data['t2'], data['t3'], uid2, uid]
            #STORED_PROCEDURE_DEMO
            change_db(
                "UPDATE Marksheet SET sub_code=?, student_id=?,  "
                "t1=?, t2=?, t3=?, cm=avg(?, ?, ?) WHERE sub_code=? AND student_id=?", dic)
            return redirect(url_for("marksheet"))
    if entity == "courses":
        values = query_db("SELECT * FROM Courses \
                            WHERE teacher_id=? AND sub_code=?",
                          [uid, uid2], one=True)
        print(values.keys())
        if request.method == 'GET':
            return render_template("modify.html", id=ID,
                                   entity="Courses", identity=values)
        if request.method == 'POST':
            data = request.form.to_dict()
            print(data.keys())
            dic = [data['teacher_id'], data['sub_code'],
                    data['Room'], uid, uid2]
            change_db(
                "UPDATE Courses SET teacher_id=?,  sub_code=?, "
                "Room=? WHERE teacher_id=? AND sub_code=?", dic)
            return redirect(url_for("courses"))


@app.route("/marksheet")
def marksheet():
    global ID
    if not session.get('logged_in'):
        return login()
    else:
        entry_list = query_db("SELECT * FROM Marksheet", [])
        return render_template("marksheet.html",  id=ID,
                               entry_list=entry_list)

@app.route("/attendance")
def attendance():
    global ID
    if not session.get('logged_in'):
        return login()
    else:
        entry_list = query_db("SELECT * FROM Attendance", [])
        return render_template("attendance.html",  id=ID,
                               entry_list=entry_list)

@app.route("/courses")
def courses():
    global ID
    if not session.get('logged_in'):
        return login()
    else:
        entry_list = query_db("SELECT * FROM Courses", [])
        return render_template("courses.html", id=ID,
                               entry_list=entry_list)

@app.route("/branch")
def branch():
    global ID
    if not session.get('logged_in'):
        return login()
    else:
        entry_list = query_db("SELECT * FROM Branch", [])
        return render_template("branch.html", id=ID,
                               entry_list=entry_list)

@app.route("/semester")
def semester():
    global ID
    if not session.get('logged_in'):
        return login()
    else:
        semester_list = []
        for i in range(8):
            semester_list.append(query_db("SELECT * FROM Subject  "
                                          "WHERE semester=?", [i + 1]))
        return render_template("semester.html", id=ID,
                               semester_list=semester_list)

@app.route('/delete/<string:entity>/<string:uid>', methods=['GET', 'POST'])
def delete(uid, entity):
    global ID
    teacher = query_db("SELECT * FROM Teacher \
                         WHERE teacher_id=?", [ID], one=True)
    if entity == "teacher":
        values = query_db("SELECT * FROM Teacher \
                            WHERE teacher_id=?", [uid], one=True)
        if request.method == 'GET':
            return render_template("delete.html", id=ID,
                                   entity="Teacher", identity=values)
        if request.method == 'POST':
            change_db("DELETE FROM Teacher WHERE teacher_id=?",
                      [uid])
            return logout()
    if entity == "student":
        values = query_db("SELECT * FROM Student \
                            WHERE student_id=?", [uid], one=True)
        if request.method == 'GET':
            return render_template("delete.html", id=ID,
                                   entity="Student",  identity=values)
        if request.method == 'POST':
            change_db("DELETE FROM Student WHERE student_id=?", [uid])
            return logout()

@app.route("/add/<string:entity>", methods=['GET', 'POST'])
def add(entity):
    global ID
    if not session.get('logged_in'):
        return login()
    else:
        if entity == "teacher":
            if request.method == 'GET':
                return render_template("add.html", id=ID, entity="Teacher")
            if request.method == 'POST':
                data = request.form.to_dict()
                dic = [data['teacher_id'], data['sub_code'],
                       data['teacher_name'],  data['phone'],
                       data['date_of_birth'],  data['password']]
                print(dic)
                change_db("INSERT INTO Teacher VALUES (?, ?, ?, ?, ?, ?)", dic)
                return logout()
        if entity == "student":
            if request.method == 'GET':
                return render_template("add.html", id=ID, entity="Student")
            if request.method == 'POST':
                data = request.form.to_dict()
                dic = [data['student_id'], data['student_name'],
                       data['date_of_birth'],  data[
                           'academic_year'],  data['branch_code'],
                       data['semester']]
                change_db("INSERT INTO Student VALUES (?, ?, ?, ?, ?, ?)", dic)
                return logout()
        if entity == "courses":
            if request.method == 'GET':
                return render_template("add.html",  id=ID,
                                       entity="Courses")
            if request.method == 'POST':
                data = request.form.to_dict()
                dic = [data['teacher_id'], data['sub_code'],
                       data['Room']]
                change_db("INSERT INTO Courses VALUES (?, ?, ?", dic)
                return logout()



if __name__ == '__main__':
    app.secret_key = os.urandom(12)

    app.run(host="0.0.0.0", port=5000, debug=False)

\end{python}

