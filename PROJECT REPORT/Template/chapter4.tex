\chapter{DESCRIPTION OF TOOLS AND TECHNOLOGIES}

\section{HTML}
Hypertext Markup Language (HTML) is the standard markup language for creating web pages and web applications. With Cascading Style Sheets (CSS) and JavaScript it forms a triad of cornerstone technologies for the World Wide Web. Web browsers receive HTML documents from a web server or from local storage and render them into multimedia web pages. HTML describes the structure of a web page semantically and originally included cues for the appearance of the document.
HTML elements are the building blocks of HTML pages. With HTML constructs, images and other objects, such as interactive forms, may be embedded into the rendered page. It provides a means to create structured documents by denoting structural semantics for text such as headings, paragraphs, lists, links, quotes and other items. HTML elements are delineated by tags, written using angle brackets. Tags such as <img /> and <input /> introduce content into the page directly. Others such as <p>...</p> surround and provide information about document text and may include other tags as sub-elements. Browsers do not display the HTML tags, but use them to interpret the content of the page.
HTML can embed programs written in a scripting language such as JavaScript which affect the behavior and content of web pages. Inclusion of CSS defines the look and layout of content. 

\section{BOOTSTRAP -A CSS FRAMEWORK}
Cascading Style Sheets (CSS) is a style sheet language used for describing the presentation of a document written in a markup language. Although most often used to set the visual style of web pages and user interfaces written in HTML and XHTML, the language can be applied to any XML document, including plain XML, SVG and XUL, and is applicable to rendering in speech, or on other media. Along with HTML and JavaScript, CSS is a cornerstone technology used by most websites to create visually engaging webpages, user interfaces for web applications, and user interfaces for many mobile applications.

CSS is designed primarily to enable the separation of presentation and content, including aspects such as the layout, colors, and fonts. This separation can improve content accessibility, provide more flexibility and control in the specification of presentation characteristics, enable multiple HTML pages to share formatting by specifying the relevant CSS in a separate .css file, and reduce complexity and repetition in the structural content.

Bootstrap is a free and open-source front-end web framework used for  designing websites and web applications. It contains HTML- and CSS-based design templates for typography, forms, buttons, navigation and other interface components, as well as optional JavaScript extensions. Unlike many web frameworks, it concerns itself with front-end development only.

Bootstrap is the second most-starred project on GitHub, with more than 111,600 stars and 51,500 forks.
\section{JAVASCRIPT}
JavaScript, often abbreviated as JS, is a high-level, interpreted programming language. It is a language which is also characterized as dynamic, weakly typed, prototype-based and multi-paradigm.

Alongside HTML and CSS, JavaScript is one of the three core technologies of the World Wide Web. JavaScript enables interactive web pages and this is an essential part of web applications. The vast majority of websites use it, and all major web browsers have a dedicated JavaScript engine to execute it.

As a multi-paradigm language, JavaScript supports event-driven, functional, and imperative (including object-oriented and prototype-based) programming styles. It has an API for working with text, arrays, dates, regular expressions, and basic manipulation of the DOM, but the language itself does not include any I/O, such as networking, storage, or graphics facilities, relying for these upon the host environment in which it is embedded.

Initially only implemented client-side in web browsers, JavaScript engines are now embedded in many other types of host software, including server-side in web servers and databases, and in non-web programs such as word processors and PDF software, and in runtime environments that make JavaScript available for writing mobile and desktop applications, including desktop widgets.

Although there are strong outward similarities between JavaScript and Java, including language name, syntax, and respective standard libraries, the two languages are distinct and differ greatly in design; JavaScript was influenced by programming languages such as Self and Scheme.

\section{PYTHON}
Python is an interpreted high-level programming language for general-purpose programming. Created by Guido van Rossum and first released in 1991, Python has a design philosophy that emphasizes code readability, notably using significant whitespace. It provides constructs that enable clear programming on both small and large scales. In July 2018, Van Rossum stepped down as the leader in the language community after 30 years.

Python features a dynamic type system and automatic memory management. It supports multiple programming paradigms, including object-oriented, imperative, functional and procedural, and has a large and comprehensive standard library.

Python interpreters are available for many operating systems. CPython, the reference implementation of Python, is open source software and has a community-based development model, as do nearly all of Python's other implementations. Python and CPython are managed by the non-profit Python Software Foundation. 


\section{SQLITE}
SQLite is a relational database management system contained in a C programming library. In contrast to many other database management systems, SQLite is not a client–server database engine. Rather, it is embedded into the end program.SQLite is ACID-compliant and implements most of the SQL standard, using a dynamically and weakly typed SQL syntax that does not guarantee the domain integrity.SQLite is a popular choice as embedded database software for local/client storage in application software such as web browsers. It is arguably the most widely deployed database engine, as it is used today by several widespread browsers, operating systems, and embedded systems (such as mobile phones), among others.SQLite has bindings to many programming languages.
Unlike client–server database management systems, the SQLite engine has no standalone processes with which the application program communicates. Instead, the SQLite library is linked in and thus becomes an integral part of the application program. The library can also be called dynamically. The application program uses SQLite's functionality through simple function calls, which reduce latency in database access: function calls within a single process are more efficient than inter-process communication. SQLite stores the entire database (definitions, tables, indices, and the data itself) as a single cross-platform file on a host machine. It implements this simple design by locking the entire database file during writing. SQLite read operations can be multitasked, though writes can only be performed sequentially.
SQLite implements most of the SQL-92 standard for SQL but it lacks some features. For example, it partially provides triggers, and it can't write to views (however it provides INSTEAD OF triggers that provide this functionality). While it provides complex queries, it still has limited ALTER TABLE function, as it can't modify or delete columns.SQLite uses an unusual type system for an SQL-compatible DBMS; instead of assigning a type to a column as in most SQL database systems, types are assigned to individual values; in language terms it is dynamically typed. Moreover, it is weakly typed in some of the same ways that Perl is: one can insert a string into an integercolumn (although SQLite will try to convert the string to an integer first, if the column's preferred type is integer). This adds flexibility to columns, especially when bound to a dynamically typed scripting language. However, the technique is not portable to other SQL products. The SQLite web site describes a "strict affinity" mode, but this feature has not yet been added.[11] However, it can be implemented with constraints like CHECK(typeof(x)='integer').


